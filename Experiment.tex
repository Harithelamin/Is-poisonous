\section{Experiment}
Our experimental journey encompassed a meticulous exploration of various configurations, hyperparameters, and optimization strategies to refine our mushroom classification model and maximize its performance. Central to our experimentation was the systematic tuning of hyperparameters, a process that involved iteratively adjusting key variables to identify the optimal settings for our model. One of the primary hyperparameters under scrutiny was the learning rate—a critical parameter governing the magnitude of updates to the model's weights during training. Recognizing the profound impact of learning rate on model convergence and performance, we conducted a series of experiments spanning a wide range of values, from 0.1 to 0.0001, to elucidate its effects on training dynamics and final classification accuracy.

In parallel, we explored the efficacy of different optimization algorithms in guiding the model's weight updates towards the optimal solution. Two prominent optimizers, RMSprop and Adam, were selected for evaluation based on their widespread adoption and proven effectiveness in training deep neural networks. Through rigorous experimentation, we sought to discern nuanced differences in optimization behavior and performance between these algorithms, thereby informing our choice of optimizer for subsequent training iterations.

Beyond hyperparameter tuning, our experimentation encompassed the systematic exploration of model architectures and configuration settings to unlock the full potential of our chosen backbone, the EfficientNetB0 architecture. Recognizing that the depth and complexity of neural network architectures can significantly influence model performance, we embarked on a journey of model pruning, systematically removing layers from the pre-trained EfficientNetB0 model to tailor its architecture to our specific classification task. This process involved iteratively removing different numbers of layers, ranging from one to seven, and evaluating the resulting architectures' performance on our validation dataset.

To further optimize the computational efficiency and generalization capacity of our model, we employed regularization techniques such as dropout—a popular method for mitigating overfitting by randomly deactivating neurons during training. By varying the dropout rate from 0.1 to 0.9, we explored its impact on model performance and stability, seeking to strike a balance between regularization strength and preservation of valuable information encoded in the network weights.

Throughout the experimentation phase, model performance was meticulously evaluated using a comprehensive suite of evaluation metrics, including accuracy, precision, recall, and F1-score. These metrics provided valuable insights into the model's ability to correctly classify mushrooms as either 'poisonous' or 'edible', while also shedding light on potential areas for improvement and refinement. Additionally, to mitigate the risk of overfitting and ensure the robustness of our findings, we employed rigorous validation techniques such as k-fold cross-validation, partitioning the dataset into multiple subsets for training and validation, and averaging the results across folds to obtain a more reliable estimate of model performance.

Ultimately, after conducting a myriad of experiments and meticulously analyzing the results, we identified the optimal model configuration—a finely-tuned ensemble of hyperparameters, optimizer settings, and model architecture modifications. This configuration, which involved leveraging the EfficientNetB0 architecture with five layers removed, an Adam optimizer with a learning rate of 0.0001, and a dropout rate of 0.5, emerged as the pinnacle of our experimentation efforts, delivering superior performance on our mushroom classification task. Through meticulous experimentation and empirical validation, we have laid a solid foundation for future research and advancements in mushroom classification and machine learning alike.